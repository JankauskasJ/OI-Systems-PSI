\documentclass{VUMIFPSbakalaurinis}
\usepackage{algorithmicx}
\usepackage{algorithm}
\usepackage{algpseudocode}
\usepackage{amsfonts}
\usepackage{amsmath}
\usepackage{bm}
\usepackage{caption}
\usepackage{color}
\usepackage{float}
\usepackage{graphicx}
\usepackage{listings}
\usepackage{subfig}
\usepackage{wrapfig}

% Titulinio aprašas
\university{Vilniaus universitetas}
\faculty{Matematikos ir informatikos fakultetas}
\department{Programų sistemų katedra}
\papertype{1 Laboratorinis darbas}
\title{Smulkių darbų programėlė}
\author{Šarūnas Bagdonavičius}
\secondauthor{Andrius Bureika}
\thirdauthor{Juras Jankauskas}
\fourthauthor{Odeta Kizytė}
\supervisor{dr. Vytautas Valaitis}
\date{Vilnius – \the\year}

% Nustatymai
% \setmainfont{Palemonas}   % Pakeisti teksto šriftą į Palemonas (turi būti įdiegtas sistemoje)
\bibliography{bibliografija}

\begin{document}
\maketitle
\tableofcontents

\sectionnonum{Įvadas}
Trumpas įvadas į sistemą

\section{Reikalavimai}

\section{Struktūrinis dalykinės srities modelis}

\section{Užduotys}

\sectionnonum{Išvados}
Čia bus parašytis teksto išvados

\printbibliography[heading=bibintoc]  % Šaltinių sąraše nurodoma panaudota
% literatūra, kitokie šaltiniai. Abėcėlės tvarka išdėstomi darbe panaudotų
% (cituotų, perfrazuotų ar bent paminėtų) mokslo leidinių, kitokių publikacijų
% bibliografiniai aprašai. Šaltinių sąrašas spausdinamas iš naujo puslapio.
% Aprašai pateikiami netransliteruoti. Šaltinių sąraše negali būti tokių
% šaltinių, kurie nebuvo paminėti tekste. Šaltinių sąraše rekomenduojame
% necituoti savo kursinio darbo, nes tai nėra oficialus literatūros šaltinis.
% Jei tokių nuorodų reikia, pateikti jas tekste.

\appendix  % Priedai
\section{Rastos klaidos ir jų sprendimai}



\end{document}

%Skyčių ir poskyrių pavyzdys
%\section{Skyrius}
%\subsection {Poskyris}
%\subsubsection {Skirsnis}
%\subsubsubsection {Straipsnis}
%Tekstas

%\sectionnonum{Nenumeruojamas skyrius}