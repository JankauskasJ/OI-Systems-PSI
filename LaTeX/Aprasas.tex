\documentclass{VUMIFPSbakalaurinis}
\usepackage{algorithmicx}
\usepackage{algorithm}
\usepackage{algpseudocode}
\usepackage{amsfonts}
\usepackage{amsmath}
\usepackage{bm}
\usepackage{caption}
\usepackage{color}
\usepackage{float}
\usepackage{graphicx}
\usepackage{listings}
\usepackage{subfig}
\usepackage{wrapfig}
\usepackage{array}
\usepackage[table]{xcolor}

% Titulinio aprašas
\university{Vilniaus universitetas}
\faculty{Matematikos ir informatikos fakultetas}
\department{Programų sistemų katedra}
\papertype{1 Laboratorinis darbas}
\title{Smulkių darbų programėlė}
\author{Šarūnas Bagdonavičius}
\secondauthor{Andrius Bureika}
\thirdauthor{Juras Jankauskas}
\fourthauthor{Odeta Kizytė}
\supervisor{dr. Vytautas Valaitis}
\date{Vilnius – \the\year}

% Nustatymai
% \setmainfont{Palemonas}   % Pakeisti teksto šriftą į Palemonas (turi būti įdiegtas sistemoje)
\bibliography{bibliografija}

\begin{document}
\maketitle
\tableofcontents

\sectionnonum{Įvadas}
Čia turės atsirasti trumpas įvadas į mūsų sistemą, ko ja siekiame ir t.t.

\section{Reikalavimai}

\section{Struktūrinis dalykinės srities modelis}

\section{Užduotys}

\sectionnonum{Išvados}
Čia reikės parašyti šio darbo išvadas, ką nuveikėme ir ko pasiekėme

\printbibliography[heading=bibintoc]  % Šaltinių sąraše nurodoma panaudota
% literatūra, kitokie šaltiniai. Abėcėlės tvarka išdėstomi darbe panaudotų
% (cituotų, perfrazuotų ar bent paminėtų) mokslo leidinių, kitokių publikacijų
% bibliografiniai aprašai. Šaltinių sąrašas spausdinamas iš naujo puslapio.
% Aprašai pateikiami netransliteruoti. Šaltinių sąraše negali būti tokių
% šaltinių, kurie nebuvo paminėti tekste. Šaltinių sąraše rekomenduojame
% necituoti savo kursinio darbo, nes tai nėra oficialus literatūros šaltinis.
% Jei tokių nuorodų reikia, pateikti jas tekste.

% Priedai
\appendix
\section{Rastos klaidos ir jų sprendimai}
\begin{table}[H]\footnotesize
	\centering
	\caption{Klaidos}
	{\rowcolors{2}{yellow!50}{yellow!40}
	\setlength{\arrayrulewidth}{0.25mm}
	{\begin{tabular}{|c|c|m{11.5cm}|} \hline
	Kodas & Terminas & Reikšmė \\
	\hline

	\hline
	\end{tabular}}
	}
	\label{tab:table example}
\end{table}

\section{Žodynas}
\begin{table}[H]\footnotesize
	\centering
	\caption{Terminų žodynas}
	{\rowcolors{2}{yellow!50}{yellow!40}
	\setlength{\arrayrulewidth}{0.25mm}
	{\begin{tabular}{|c|c|m{11.5cm}|} \hline
	Kodas & Terminas & Reikšmė \\
	\hline
	T1 & Naudotojas & Fizinis asmuo, kuris turi paskyrą aplikacijoje, kurios dėka gali pateikti pagalbos prašymą ar pagalbą suteikt \\
	T2 & Pagalbos prašantysis & Mobilios aplikacijos naudotojas, kuris išsikviečia pagalbą \\
	T3 & Pagalbos teikėjas & Mobilios aplikacijos naudotojas, kuris  priima pagalbos prašymą ir atlieka darbą \\
	T4 & Administratorius & Fizinis asmuo, turintis išskirtines teises sistemoje, prižiūri sąžiningą darbų atlikimą ir atsiskaitymą. Nesilaikančius taisyklių vartotojus gali užblokuoti \\
	T5 & Pagalbos prašymas & Užklausa aplikacijoje, kurią naudotojas pateikia sistemai pagal sistemoje nurodytą procedūrą, ir kurioje tiesiogiai nurodomas norimos pagalbos pobūdis \\
	T6 & Atsiskaitymas & Pagalbos prašytojo pervedama taškų/kreditų suma už atlikta darbą/paslaugą \\
	\hline
	\end{tabular}}
	}
	\label{tab:table example}
\end{table}
	


\end{document}

%Skyčių ir poskyrių pavyzdys
%\section{Skyrius}
%\subsection {Poskyris}
%\subsubsection {Skirsnis}
%\subsubsubsection {Straipsnis}
%Tekstas

%\sectionnonum{Nenumeruojamas skyrius}